\documentclass[11pt,]{article}
\usepackage[left=1in,top=1in,right=1in,bottom=1in]{geometry}
\newcommand*{\authorfont}{\fontfamily{phv}\selectfont}
\usepackage[]{mathpazo}


  \usepackage[T1]{fontenc}
  \usepackage[utf8]{inputenc}




\usepackage{abstract}
\renewcommand{\abstractname}{}    % clear the title
\renewcommand{\absnamepos}{empty} % originally center

\renewenvironment{abstract}
 {{%
    \setlength{\leftmargin}{0mm}
    \setlength{\rightmargin}{\leftmargin}%
  }%
  \relax}
 {\endlist}

\makeatletter
\def\@maketitle{%
  \newpage
%  \null
%  \vskip 2em%
%  \begin{center}%
  \let \footnote \thanks
    {\fontsize{18}{20}\selectfont\raggedright  \setlength{\parindent}{0pt} \@title \par}%
}
%\fi
\makeatother




\setcounter{secnumdepth}{0}


\usepackage{graphicx,grffile}
\makeatletter
\def\maxwidth{\ifdim\Gin@nat@width>\linewidth\linewidth\else\Gin@nat@width\fi}
\def\maxheight{\ifdim\Gin@nat@height>\textheight\textheight\else\Gin@nat@height\fi}
\makeatother
% Scale images if necessary, so that they will not overflow the page
% margins by default, and it is still possible to overwrite the defaults
% using explicit options in \includegraphics[width, height, ...]{}
\setkeys{Gin}{width=\maxwidth,height=\maxheight,keepaspectratio}


\title{A Pandoc Markdown Article Starter and Template \thanks{This article was written with tremendous help from Zhiyang Jia.}  }



\author{\Large Henning Holgersen\vspace{0.05in} \newline\normalsize\emph{Statistics Norway}  }


\date{}

\usepackage{titlesec}

\titleformat*{\section}{\normalsize\bfseries}
\titleformat*{\subsection}{\normalsize\itshape}
\titleformat*{\subsubsection}{\normalsize\itshape}
\titleformat*{\paragraph}{\normalsize\itshape}
\titleformat*{\subparagraph}{\normalsize\itshape}





\newtheorem{hypothesis}{Hypothesis}
\usepackage{setspace}


% set default figure placement to htbp
\makeatletter
\def\fps@figure{htbp}
\makeatother

\usepackage{hyperref}
\usepackage{amsmath}
\usepackage{booktabs}
\usepackage{caption}
\usepackage{longtable}

% move the hyperref stuff down here, after header-includes, to allow for - \usepackage{hyperref}

\makeatletter
\@ifpackageloaded{hyperref}{}{%
\ifxetex
  \PassOptionsToPackage{hyphens}{url}\usepackage[setpagesize=false, % page size defined by xetex
              unicode=false, % unicode breaks when used with xetex
              xetex]{hyperref}
\else
  \PassOptionsToPackage{hyphens}{url}\usepackage[draft,unicode=true]{hyperref}
\fi
}

\@ifpackageloaded{color}{
    \PassOptionsToPackage{usenames,dvipsnames}{color}
}{%
    \usepackage[usenames,dvipsnames]{color}
}
\makeatother
\hypersetup{breaklinks=true,
            bookmarks=true,
            pdfauthor={Henning Holgersen (Statistics Norway)},
             pdfkeywords = {employment, covid-19},  
            pdftitle={A Pandoc Markdown Article Starter and Template},
            colorlinks=true,
            citecolor=blue,
            urlcolor=blue,
            linkcolor=magenta,
            pdfborder={0 0 0}}
\urlstyle{same}  % don't use monospace font for urls

% Add an option for endnotes. -----


% add tightlist ----------
\providecommand{\tightlist}{%
\setlength{\itemsep}{0pt}\setlength{\parskip}{0pt}}

% add some other packages ----------

% \usepackage{multicol}
% This should regulate where figures float
% See: https://tex.stackexchange.com/questions/2275/keeping-tables-figures-close-to-where-they-are-mentioned
\usepackage[section]{placeins}


\begin{document}
	
% \pagenumbering{arabic}% resets `page` counter to 1 
%
% \maketitle

{% \usefont{T1}{pnc}{m}{n}
\setlength{\parindent}{0pt}
\thispagestyle{plain}
{\fontsize{18}{20}\selectfont\raggedright 
\maketitle  % title \par  

}

{
   \vskip 13.5pt\relax \normalsize\fontsize{11}{12} 
\textbf{\authorfont Henning Holgersen} \hskip 15pt \emph{\small Statistics Norway}   

}

}








\begin{abstract}

    \hbox{\vrule height .2pt width 39.14pc}

    \vskip 8.5pt % \small 

\noindent The COVID-19 pandemic has forced great societal changes, including
forcing many to work remotely (work from home) in an effort to increase
social distancing. The ability to work from home has long been
considered a perk, but only a few surveys have covered this. We
demonstrate two alternative approaches to estimating the opportunities
of working remotely, and compare the results with existing survey data.


\vskip 8.5pt \noindent \emph{Keywords}: employment, covid-19 \par

    \hbox{\vrule height .2pt width 39.14pc}



\end{abstract}


\vskip -8.5pt


 % removetitleabstract

\noindent  

\hypertarget{introduction}{%
\subsection{Introduction}\label{introduction}}

The COVID-19 pandemic has propelled the question of working from home
into the spotlight, as a matter of enforcing social distancing. The
crisis has forced us to learn more about what possibilities for working
remotely already exists, including questions about what percentage of
the workforce are able to work from home, and what types of jobs can be
done from home.

In 2018, the norwegian LFS in asked respondents if they were able and
allowed to work from home at times. After the pandemic outbreak, at
least one {[}TODO: REF{]} survey has employees about their new work
situation with respect to working from home.

Social media has been quick to argue that the people who are able to
work at home are already privilidged, highly educated and highly paid,
while those who still have to go to work are more often in low-paid but
critical roles such as drivers and grocers.

In an alternative approach focusing on existing occupation
classifications, we estimate the share of the norwegian workforce able
to work remotely by combining register-based labor statistics, annotated
job advertisements, and the official documentation of the ISCO-08
standard which is used both in labor statistics and in the job ads. Most
of the data about the norwegian workforce is based on register data, and
the information we have about what people do at work are based on
occupational codes following the ISCO standard. Labor statistics are
currently published following the ISCO-08 standard (ILO 2016).

ISCO-08 is a taxonomy aimed at classifying jobs into ``occupations'',
which are relatable to the reader. These occupations do not say directly
wether the job can be performed from home. By evaluating the ISCO-08
documentation, we are able to categorize each occupation as likely or
unlikely to be remote-friendly. This classification is evaluated against
job ads published by the norwegian welfare administration (NAV) between
january 2012 and march 2019, in which there are mentions of remote
possibilities.

We find that approximately 34\% of norwegian jobs can be performed from
home.

\hypertarget{methodology}{%
\subsection{Methodology}\label{methodology}}

In order annotate the ISCO-08 documentation and provide a assertion of
wether each occupation is likely able to be performed from home, we
created a public labeling job through Amazon Mechanical Turk (AWS,
n.d.). Each occupation was presented together with a brief description,
and the respondent was asked to evaluate wether it was likely that the
job could be performed from a private home for a period of up to two
weeks. We consider that two weeks is enough to argue that the job can
\emph{mainly} be performed from home. Two weeks is an arbitrary line to
draw, but the neuances of this timeframe is unlikely to have any
significant effect on the result.

The exact question formulation was ``Can this type of job likely be
performed from a home office?'', and an example of a job description
could be:

\begin{quote}
Social work and counselling professionals provide advice and guidance to
individuals, families, groups, communities and organizations in response
to social and personal difficulties. They assist clients to develop
skills and access resources and support services needed to respond to
issues arising from unemployment, poverty, disability, addiction,
criminal and delinquent behaviour, marital and other problems.
\end{quote}

The alternatives were ``Yes'', ``No'' and ``Unknown'', which were
provided with the following description: - Yes: This job can be
performed primarily from an office in a private home - No: Substatantial
parts of this job must be performed outside the employees home -
Unknown: There is not enough information to decide

The last option, ``Unknown'' was added in order to reduce arbitrary
responses to uninformative occupation descriptions.

Since the job was on Mechanical Turk, there respondents are not subject
matter experts, and likely reside in different countries. This adds to
the importance of obtaining more than one label per occupation, but the
number of labels does not correct for possible cultural differences - it
is possible that some jobs that cannot be performed remotely in other
countries can be performed remotely in Norway. We should consider the
annotations as ``international'', which is also true for the ISCO-08
standard itself.

In order to evaluate the annotations from Mechanical Turk, we use job
ads from the norwegian welfare administration. Since the possibility to
work from home is a perk for many, some employers mention it in their
job ads in order to attract candidates. We search the texts for mentions
of ``hjemmekontor'' and ``heimekontor'', two distinctive words unlikely
to mean anything other than the possibility of working from home. Since
far from every employer advertise this possibility, it is difficult to
say anything about the total number of remote-friendly jobs from these
ads. It does however say something important about the relative
frequency of remote-friendly jobs across broader occupational groups,
which we can use to validate the results from Mechanical Turk.

\hypertarget{results}{%
\subsection{Results}\label{results}}

\hypertarget{total-remote-percentage}{%
\subsubsection{Total remote-percentage}\label{total-remote-percentage}}

From the Mechanical Turk annotations we see that 28 per cent of the
occupations can likely be performed from home. Taken together with the
labor statistics per occupation, we find that 875 344 occupations, 36
per cent of the workforce, are likely able to work from home.

\hypertarget{validating-results-against-job-ads}{%
\subsubsection{Validating results against
job-ads}\label{validating-results-against-job-ads}}

The job ads can not be compared directly to the annotated ISCO-08 data,
but one of the comparisons we are able to do is the relative frequency
of remote possibilities across broader occupational groups. Some job ads
mention remote work, and we can expect that remote possibilities are
mentioned more often within groups where remote possibilities are an
option. We can compare this frequency with the number of remote-friendly
occupations within each broader group. We also compute the composition
of remote-friendly jobs across the occupational groups, giving us a
measure that is directly comparable.

\captionsetup[table]{labelformat=empty,skip=1pt}
\begin{longtable}{lrrr}
\toprule
ISCO1 & relative\_prob\_annotations & relative\_prob\_ads & pctpts\_diff \\ 
\midrule
Managers & $14.2\%$ & $11.6\%$ & $2.6\%$ \\ 
Academics & $39.2\%$ & $41.9\%$ & $-2.7\%$ \\ 
Technicians and associate professionals & $19.2\%$ & $26.9\%$ & $-7.8\%$ \\ 
clerical support workers & $16.7\%$ & $5.3\%$ & $11.4\%$ \\ 
Service and sales workers & $4.2\%$ & $12.7\%$ & $-8.5\%$ \\ 
Skilled agricultural, forestry and fishery workers & $0.0\%$ & $0.1\%$ & $-0.1\%$ \\ 
Craft and related trades workers & $6.7\%$ & $1.3\%$ & $5.4\%$ \\ 
Plant and machine operators and assemblers & $0.0\%$ & $0.1\%$ & $-0.1\%$ \\ 
Elementary Occupations & $0.0\%$ & $0.1\%$ & $-0.1\%$ \\ 
\bottomrule
\end{longtable}

The noteable discrepancies for the groups ``Service and sales workers''
and ``Clearical support workers'' have a strange symmetry. These two
groups share a lot of similarities, but one of the differentiators are
that sales jobs are explicitly customer facing, while clerical support
workers typically work in a backoffice position. Intuitively this would
suggest that somehow the annotations are correct while the job ads are
incorrect, but many explanations exist.

The correlation between the two measures are strong, but in order to
evaluate it statistically we would benefit from more detailed data. One
approach is to use a logistic regression explaining the annotation of an
occupation (a binary label) with the ratio of mentions of
``hjemmekontor'' in the ad.

\begin{verbatim}
## [1] "ISCO"               "wfh"                "confidence"        
## [4] "ISCO1"              "wfh_dummy"          "wfh_prob"          
## [7] "number_of_ads"      "number_of_mentions" "relative_prob_ads"
\end{verbatim}

\begin{verbatim}
## # A tibble: 2 x 5
##   term        estimate std.error statistic  p.value
##   <chr>          <dbl>     <dbl>     <dbl>    <dbl>
## 1 (Intercept)   -0.952     0.120     -7.94 2.00e-15
## 2 wfh_prob      31.4      15.9        1.97 4.86e- 2
\end{verbatim}

The noteable discrepancies for the groups ``Service and sales workers''
and ``Clearical support workers'' have a strange symmetry. These two
groups share a lot of similarities, but one of the differentiators are
that sales jobs are explicitly customer facing, while clerical support
workers typically work in a backoffice position. Intuitively this would
suggest that somehow the annotations are correct while the job ads are
incorrect, but many explanations exist.

The correlation between the two measures are strong, but in order to
evaluate it statistically we would benefit from more detailed data. One
approach is to use a logistic regression explaining the annotation of an
occupation (a binary label) with the ratio of mentions of
``hjemmekontor'' in the ad.

\begin{verbatim}
## [1] "ISCO"               "wfh"                "confidence"        
## [4] "ISCO1"              "wfh_dummy"          "wfh_prob"          
## [7] "number_of_ads"      "number_of_mentions" "relative_prob_ads"
\end{verbatim}

\captionsetup[table]{labelformat=empty,skip=1pt}
\begin{longtable}{lrrrr}
\toprule
term & estimate & std.error & statistic & p.value \\ 
\midrule
(Intercept) & -1 & 0.1 & -8 & $0.0000$ \\ 
wfh\_prob & 31 & 15.9 & 2 & $0.0486$ \\ 
\bottomrule
\end{longtable}

Menitoning ``hjemmekontor'' in the add is significantly correlated with
the annotation, if only within an inch of the usual 5 \% significance
threshold.

Given the spuriousness of the data sources, the results seem fairly well
aligned.

\hypertarget{analyzing-the-remote-occupations}{%
\subsubsection{Analyzing the remote
occupations}\label{analyzing-the-remote-occupations}}

Splitting the annotated data into occupational groups, we can see what
percentage of occupations are remote-friendly across occupational
groups:

\captionsetup[table]{labelformat=empty,skip=1pt}
\begin{longtable}{llrrr}
\toprule
& & \multicolumn{2}{c}{Number of jobs} & \\ 
 \cmidrule(lr){3-4}
 & Occupational Group & Total & Remote-friendly & Percent remote friendly \\ 
\midrule
 &  & Managers & 183412 & 94959 \\ 
$51,8\%$ &  &  & Academics & 576136 \\ 
297256 & $51,6\%$ &  &  & Technicians and associate professionals \\ 
373065 & 158504 & $42,5\%$ &  &  \\ 
clerical support workers & 169230 & 112842 & $66,7\%$ &  \\ 
 & Service and sales workers & 570761 & 165493 & $29,0\%$ \\ 
 &  & Skilled agricultural, forestry and fishery workers & 21631 & 0 \\ 
$0,0\%$ &  &  & Craft and related trades workers & 219658 \\ 
46290 & $21,1\%$ &  &  & Plant and machine operators and assemblers \\ 
\bottomrule
\end{longtable}

For each broad occupational group, the share of jobs that can be
performed remotely varies from 0 to 67 per cent. Academics and managers
are both groups where more than half of the employees can be done
remotely, but an even higher share of clerical support workers are
likely able to do their jobs from home.

This bears strong resemblance to the table comparing the relative
percentages of ads and annotations, but now with absolute numbers.
Managers and academics do indeed often have jobs that can be performed
from home, but this is even more the case for clerical support workers
where two thirds theoretically could be working from home.

The geographic location of jobs have been a point of interest for years,
amid both pressure for workers to centralize and specialize, and fears
of increased inequality between cities and rural areas. Cities have a
higher share of remote-friendly jobs, which may be fortunate given the
need for social distancing.

\includegraphics{figs/unnamed-chunk-6.pdf}

The pattern looks clear, especially in the area surrounding Oslo but
also the other cities like Bergen, Trondheim and Stavanger seems to
stand out on the map.

By introducing a measure of urbanness, we can analyze the relationship
more formally. We use population per square km as a proxy for urbanness.

We can see a clear correlation between ``urbanness'', or population pr
km\(^2\), and the availability of remote-friendly jobs.

\includegraphics{figs/unnamed-chunk-7.pdf}

In the simple linear model we have fitted, Person pr km\(^2\) is clearly
significant.

\captionsetup[table]{labelformat=empty,skip=1pt}
\begin{longtable}{lrrrr}
\toprule
term & estimate & std.error & statistic & p.value \\ 
\midrule
(Intercept) & 32.08 & 0.1488 & 216 & $0.0000$ \\ 
ppkm & 0.01 & 0.0008 & 13 & $0.0000$ \\ 
\bottomrule
\end{longtable}

\hypertarget{similar-work}{%
\subsection{Similar work}\label{similar-work}}

The norwegian labor force survey has asked about remote-possibilities,
covered in a report by FAFO (Nergaard 2018). The question was wether the
respondent had the opportunity to work from home at times, which is not
to say that the job could be performed remotely in its entirety, and
neither to say that those who weren't given the opportunity couldn't
have worked from home if they had the option.

Nevertheless the results from this survey was broadly similar to these.
In the survey, 71 per cent of managers responded that they had
opportunity to work from home at times, much higher than the results
here. This difference is likely attributable to the distinction
mentioned above: Being able to perform \emph{some} part of your job
remotely does not mean that the job can be performed \emph{primarily}
remotely.

A recent working paper {[}NBER:1{]} explores the same question by using
the O*net ontology - an approach which directly inspired this paper.
Also here, the findings are generally similar, even though the paper is
based on US data. The O*net ontology is empirically derived, lending it
credibility. It is also very rich in how it describes occupations and
their content.

\hypertarget{further-work}{%
\subsection{Further work}\label{further-work}}

The sudden question of remote work highlights the need to expand our
knowledge of occupations and their contents. For National Statistical
Institutions in Europe, the most natural option would be to expand the
existing ESCO ontology with this data. The results we have found also
suggest that alternative approaches to collecting such information is
feasible.

\hypertarget{refs}{}
\leavevmode\hypertarget{ref-MT}{}%
AWS. n.d. ``Amazon Mechanical Turk.'' \url{https://www.mturk.com/}.

\leavevmode\hypertarget{ref-ISCO:08}{}%
ILO. 2016. ``ISCO - International Standard Classification of
Occupations.''
\url{https://www.ilo.org/public/english/bureau/stat/isco/isco08/index.htm}.

\leavevmode\hypertarget{ref-FAFO:HK}{}%
Nergaard. 2018. ``Fleksibel Arbeidstid: En Analyse Av Ordninger I Norsk
Arbeidsliv.'' \url{https://www.fafo.no/images/pub/2018/20664.pdf}.

\newpage
\singlespacing 
\end{document}