\documentclass[11pt,]{article}
\usepackage[left=1in,top=1in,right=1in,bottom=1in]{geometry}
\newcommand*{\authorfont}{\fontfamily{phv}\selectfont}
\usepackage[]{mathpazo}


  \usepackage[T1]{fontenc}
  \usepackage[utf8]{inputenc}




\usepackage{abstract}
\renewcommand{\abstractname}{}    % clear the title
\renewcommand{\absnamepos}{empty} % originally center

\renewenvironment{abstract}
 {{%
    \setlength{\leftmargin}{0mm}
    \setlength{\rightmargin}{\leftmargin}%
  }%
  \relax}
 {\endlist}

\makeatletter
\def\@maketitle{%
  \newpage
%  \null
%  \vskip 2em%
%  \begin{center}%
  \let \footnote \thanks
    {\fontsize{18}{20}\selectfont\raggedright  \setlength{\parindent}{0pt} \@title \par}%
}
%\fi
\makeatother




\setcounter{secnumdepth}{0}


\usepackage{graphicx,grffile}
\makeatletter
\def\maxwidth{\ifdim\Gin@nat@width>\linewidth\linewidth\else\Gin@nat@width\fi}
\def\maxheight{\ifdim\Gin@nat@height>\textheight\textheight\else\Gin@nat@height\fi}
\makeatother
% Scale images if necessary, so that they will not overflow the page
% margins by default, and it is still possible to overwrite the defaults
% using explicit options in \includegraphics[width, height, ...]{}
\setkeys{Gin}{width=\maxwidth,height=\maxheight,keepaspectratio}


\title{Working from home: estimations without surveys \thanks{TBD}  }



\author{\Large Henning Holgersen\vspace{0.05in} \newline\normalsize\emph{Statistics Norway}   \and \Large Zhiyang Jia\vspace{0.05in} \newline\normalsize\emph{Statistics Norway}   \and \Large Simen Svenkerud\vspace{0.05in} \newline\normalsize\emph{Statistics Norway}  }


\date{}

\usepackage{titlesec}

\titleformat*{\section}{\normalsize\bfseries}
\titleformat*{\subsection}{\normalsize\itshape}
\titleformat*{\subsubsection}{\normalsize\itshape}
\titleformat*{\paragraph}{\normalsize\itshape}
\titleformat*{\subparagraph}{\normalsize\itshape}


\usepackage{natbib}
\bibliographystyle{apsr}
\usepackage[strings]{underscore} % protect underscores in most circumstances



\newtheorem{hypothesis}{Hypothesis}
\usepackage{setspace}


% set default figure placement to htbp
\makeatletter
\def\fps@figure{htbp}
\makeatother

\usepackage{hyperref}
\usepackage{}

% move the hyperref stuff down here, after header-includes, to allow for - \usepackage{hyperref}

\makeatletter
\@ifpackageloaded{hyperref}{}{%
\ifxetex
  \PassOptionsToPackage{hyphens}{url}\usepackage[setpagesize=false, % page size defined by xetex
              unicode=false, % unicode breaks when used with xetex
              xetex]{hyperref}
\else
  \PassOptionsToPackage{hyphens}{url}\usepackage[draft,unicode=true]{hyperref}
\fi
}

\@ifpackageloaded{color}{
    \PassOptionsToPackage{usenames,dvipsnames}{color}
}{%
    \usepackage[usenames,dvipsnames]{color}
}
\makeatother
\hypersetup{breaklinks=true,
            bookmarks=true,
            pdfauthor={Henning Holgersen (Statistics Norway) and Zhiyang Jia (Statistics Norway) and Simen Svenkerud (Statistics Norway)},
             pdfkeywords = {Covid-19, Working from home, Job advertisements, Unconventional data,
Norway},  
            pdftitle={Working from home: estimations without surveys},
            colorlinks=true,
            citecolor=blue,
            urlcolor=blue,
            linkcolor=magenta,
            pdfborder={0 0 0}}
\urlstyle{same}  % don't use monospace font for urls

% Add an option for endnotes. -----


% add tightlist ----------
\providecommand{\tightlist}{%
\setlength{\itemsep}{0pt}\setlength{\parskip}{0pt}}

% add some other packages ----------

% \usepackage{multicol}
% This should regulate where figures float
% See: https://tex.stackexchange.com/questions/2275/keeping-tables-figures-close-to-where-they-are-mentioned
\usepackage[section]{placeins}


\begin{document}
	
% \pagenumbering{arabic}% resets `page` counter to 1 
%
% \maketitle

{% \usefont{T1}{pnc}{m}{n}
\setlength{\parindent}{0pt}
\thispagestyle{plain}
{\fontsize{18}{20}\selectfont\raggedright 
\maketitle  % title \par  

}

{
   \vskip 13.5pt\relax \normalsize\fontsize{11}{12} 
\textbf{\authorfont Henning Holgersen} \hskip 15pt \emph{\small Statistics Norway}   \par \textbf{\authorfont Zhiyang Jia} \hskip 15pt \emph{\small Statistics Norway}   \par \textbf{\authorfont Simen Svenkerud} \hskip 15pt \emph{\small Statistics Norway}   

}

}








\begin{abstract}

    \hbox{\vrule height .2pt width 39.14pc}

    \vskip 8.5pt % \small 

\noindent The COVID-19 crisis has forced great societal changes, including forcing
many to work remotely (work from home) in an effort to increase social
distancing. The ability to work from home has long been considered a
perk, but we have few estimates of how many are actually able to work
from home. Social media has been quick to argue that the people who are
able to work at home are already privileged, highly educated and highly
paid, while those who still have to go to work are more often in
low-paid but critical roles such as drivers and grocers. This paper
attempts to estimate the share of the Norwegian workforce able to work
remotely by combining register-based labor statistics, annotated job ads
and the official documentation of the ISCO-08 standard which is used
both in labor statistics and in the job ads. We find that approximately
36 \% of Norwegian jobs can be performed at home.


\vskip 8.5pt \noindent \emph{Keywords}: Covid-19, Working from home, Job advertisements, Unconventional data,
Norway \par

    \hbox{\vrule height .2pt width 39.14pc}



\end{abstract}


\vskip -8.5pt


 % removetitleabstract

\noindent  

\hypertarget{introduction}{%
\section{Introduction}\label{introduction}}

Covid-19 pandemic hit the world hard and unprepared. In a study of the
Spanish flu, \citep{Hatchett7582} shows that non-pharmaceutical
interventions known as ``social distancing'' during a pandemic can
significantly reduce the disease transmission and lower both the peak
and cumulative excess mortality. Learning from the historical lessons,
many countries, including Norway, implemented measures to limit physical
contacts between people. Encouraging working remotely (working from
home) is one important part of these measures. However, concerns are
raised quickly among experts and ordinary people on both its
effectiveness and impact on social fairness. They argue that such policy
will help only those who are high educated, highly paid and have
comfortable jobs. The people who are much more susceptible benefit
little from such measures. These concerns are well founded, but they
ignored the externality of this measure. In fact, the measure will
reduce the risk of infection indirectly for those who cannot work from
home as well, since it will reduce both the potential infection risk and
frequency of exposures to such risk. Here, the prevalence of remote
feasible jobs matters. If there is only a very small fraction of working
force can work from home, the effect may be ignorable to people others
than those who work from home. Large prevalence of such jobs will imply
higher effectiveness and lower negative consequence on social equality.
However, there is very limited knowledge on prevalence of remote
feasible jobs. In this paper, we try to answer the question: who and how
many can work from home in Norway. Our analysis is based on three
different types of data: a) the information of tasks described in the
ISCO-08 standard \citep{ILO12}. b) Job advertisements published by the
Norwegian welfare administration (NAV) between January 2012 and march
2019, in which there are mentions of remote possibilities. c) official
statistics on occupational employment published by Statistics Norway.
The feasibility of an occupation is evaluated using ISCO descriptions of
tasks to be performed. Detailed explanation of such classification is
explained in the following in section sec:methodology {[}!TODO{]}. The
classifications are then linked to the job advertisement data for
consistency check and potential bias connection. Combing the feasibility
classifications with occupational employment statistics, we then obtain
information on the prevalence of remote feasible jobs and some
characteristics of workers with such jobs. With help of the employment
register, this can be done not only on national level but also on
smaller geographical areas: for example, on municipalities.

We find that approximately 39 \% of Norwegian jobs can be performed at
home. As we expected, remote friendly jobs are often paid better than
non-remote friendly jobs. There are large geographical differences in
the prevalence of remote friendly jobs. There are larger share of remote
friendly jobs in urban areas than rural areas. To some extent, this is a
good news given that urban areas often facing a large challenges in
containing the spread of Covid-19 given their high population density.

The method we applied here is similar to a very recent study by
\citep{Dingel2020}. They explores the same question in the United States
using the Occupational Information Network (O*net) surveys covering
``work context'' and ``generalized work activities''. Unlike their
study, we reply on much less conventional types of data: the description
of tasks and contexts from job advertisements, which in turn requires
new techniques for information extraction: manual labelling and natural
language processing (NLP) techniques. Our study shows that combining
conventional and unconventional data sources can lead to novel
information that are useful for both official statistics, research and
social policy making.

There were two surveys in norway in which some remote work feasibility
is asked: The Norwegian labor force survey and a recent survey by the
Norwegian Institute of Transport Economics(TØI). The Norwegian labor
force survey has asked about remote-possibilities, covered in a report
by \citep{Nergaard2018}. The question was whether the respondent had the
opportunity to work from home at times, which is not to say that the job
could be performed remotely in its entirety, and neither to say that
those who weren't given the opportunity couldn't have worked from home
if they had the option. The survey by TØI is designed specially for the
Covid-19 situation with the main focus on the effective of remote work
\citep{Nordbakke2020}. While the survey does provide an overall estimate
of the prevalence of remote jobs, information asked is rather limited.
Neither occupational nor geographical aspects of the jobs are collected.
Nevertheless the results we get were broadly similar to these surveys,
while our analysis provides much more detailed information.

\hypertarget{method}{%
\section{Method}\label{method}}

\hypertarget{the-international-standard-classification}{%
\subsection{The International Standard
Classification}\label{the-international-standard-classification}}

The ISCO standard organizes jobs into a set of groups according to the
tasks and duties undertaken in the job. Using the detailed task
descriptions listed in the ISCO-08 documentation, we try to provide a
assertion of whether an occupation is likely able to be performed from
home. To do this, we created a public labeling job through Amazon
Mechanical Turk \citep{Turk2020}. Each occupation was presented together
with a brief description. The exact question formulation was ``Can this
type of job likely be performed from a home office?'', and an example of
a job description could be:

\begin{quote}
Social work and counselling professionals provide advice and guidance to
individuals, families, groups, communities and organizations in response
to social and personal difficulties. They assist clients to develop
skills and access resources and support services needed to respond to
issues arising from unemployment, poverty, disability, addiction,
criminal and delinquent behaviour, marital and other problems.
\end{quote}

The respondent was asked to evaluate wether it was likely that the job
could be performed primarily from a private home. The alternatives were
``Yes'', ``No'' and ``Unknown'', which were provided with the following
description:

\begin{enumerate}
\def\labelenumi{\arabic{enumi}.}
\item
  \emph{Yes: This job can be performed primarily from an office in a
  private home}
\item
  \emph{No: Substatantial parts of this job must be performed outside
  the employees home}
\item
  \emph{Unknown: There is not enough information to decide}
\end{enumerate}

In order to reduce the serendipity in the labels, we acquired five
labels from different respondents for each occupation, and we provided
an \texttt{uncertain} option in addition to the \texttt{yes/no} options
in order to reduce arbitrary responses to uninformative occupation
descriptions. The final labels include an uncertainty measure which
shows that some of the occupations were evaluated differently by
different annotators, but no occupation was given a final label of
``Unknown'' which means we can treat the remote-friendly annotation as a
binary variable.

Since the job was on Mechanical Turk, there respondents were not subject
matter experts, and likely reside in different countries. This adds to
the importance of obtaining more than one label per occupation, but the
number of labels does not correct for possible cultural differences - it
is possible that some jobs that cannot be performed remotely in other
countries can be performed remotely in Norway. We should consider the
annotations as \texttt{international}, which is also true for the
ISCO-08 standard itself.

\hypertarget{consistency-check-using-the-job-announcements-data}{%
\subsection{Consistency check using the job announcements
data}\label{consistency-check-using-the-job-announcements-data}}

In order to evaluate the annotations from Mechanical Turk, we use job
advertisements from the Norwegian welfare administration (NAV). The job
advertisements have been published as open data by NAV, and contain the
text, title, employer information, and annotations made by subject
matter experts at NAV including the occupational code (ISCO) of the job.
The dataset covers January 2002 through march 2019 (a total of 2.6
million ads), but due to changes in the ISCO structure (a switch from
ISCO-98 to ISCO-08 in 2012), we only use ads from 2012 and onward.
Furthermore, the volume of NAV ads increased sharply in 2018 due to new
sources of data. From 2018 the NAV data is close to complete in covering
formally advertised jobs in Norway. Some shops still only advertise in
their shop window, but the vast majority of jobs posted online are now
also posted on NAV.

Because the possibility to work from home is a perk for many, some
employers mention it in their job ads in order to attract candidates. We
search the texts for mentions of \texttt{hjemmekontor} and
\texttt{heimekontor}, two distinctive words unlikely to mean anything
other than the possibility of working from home. Since far from every
employer advertise this possibility, it is difficult to say anything
about the total number of remote-friendly jobs from these ads. It may
however say something important about the relative frequency of
remote-friendly jobs across broader occupational groups, which we can
use to validate the results from Mechanical Turk.

We have also tried to combine the two data sources in a more rigorous
way to correct possible bias in the annotation labels. However, this can
only be done under rather strong assumptions on employee's behavior.
Detailed discussion can be found in the appendix.

\hypertarget{evaluate-remote-feasibility}{%
\subsection{Evaluate Remote
feasibility}\label{evaluate-remote-feasibility}}

With the estimated remote feasibility for each occupation, we use the
Norwegian labor market data from Statistics Norway (SSB) to evaluate the
remote feasibility in Norway. The data from SSB covers employment,
earnings and demographics. Employment data is register-based statistics
which comes from two tables, covering the number of employed in Norway,
divided by municipality and occupation respectively. Earnings data is
also register based, and shows average monthly earnings by occupation.
The demographics data shows population and density per municipality.

A note on nomenclature: For brevity, we sometimes refer to ``remote''
occupations rather than ``occupations that can be performed remotely''.
We use the terms interchangeably, always referring to occupations that
can be performed from home. This does not mean that such employees in
actuality work from home either permanently or occasionally.

\hypertarget{remote-feasibility-using-the-annotation-data}{%
\subsection{Remote feasibility using the annotation
data}\label{remote-feasibility-using-the-annotation-data}}

From the Mechanical Turk annotations, around 28 per cent of the
occupations can likely be performed from home. Combining the annotations
with labor statistics published per occupation, we find that 875 344
wage employers, 36 per cent of the workforce, are likely able to work
from home.

From the Mechanical Turk annotations we see that 34 per cent of the
occupations can likely be performed from home. Taken together with the
labor statistics per occupation, we find that 974 817 jobs, 39 per cent
of the workforce, are likely able to work from home.

From the Mechanical Turk annotations, around 34 per cent of the
occupations can likely be performed from home. Taken together with the
labor statistics per occupation, we find that 974 817 wage employees, 39
per cent of the workforce, are likely able to work from home.

Splitting the annotated data into occupational groups, we estimate what
percentage of occupations are remote-friendly across occupational
groups. The results are presented in Table tab:Percent\_occupation\}
{[}!TODO{]}. For each broad occupational group, the share of jobs that
can be performed remotely varies from 0 to 67 per cent. Academics and
managers are both groups where more than half of the employees can be
done remotely, but an even higher share of clerical support workers are
likely able to do their jobs from home.

\begin{verbatim}
## \captionsetup[table]{labelformat=empty,skip=1pt}
## \begin{longtable}{lrrr}
## \toprule
## & \multicolumn{2}{c}{Number of jobs} & \\ 
##  \cmidrule(lr){2-3}
## Occupational Group & Total & Remote-friendly & Percent remote friendly \\ 
## \midrule
## Managers & $222 678$ & $148 891$ & $66,9\%$ \\ 
## Academics & $652 356$ & $320 020$ & $49,1\%$ \\ 
## Technicians and associate professionals & $374 858$ & $151 153$ & $40,3\%$ \\ 
## clerical support workers & $169 230$ & $105 469$ & $62,3\%$ \\ 
## Service and sales workers & $573 415$ & $195 888$ & $34,2\%$ \\ 
## Skilled agricultural, forestry and fishery workers & $21 631$ & $3 735$ & $17,3\%$ \\ 
## Craft and related trades workers & $219 843$ & $36 196$ & $16,5\%$ \\ 
## Plant and machine operators and assemblers & $163 197$ & $11 207$ & $6,9\%$ \\ 
## Elementary Occupations & $134 400$ & $2 259$ & $1,7\%$ \\ 
## \bottomrule
## \end{longtable}
\end{verbatim}

In general, occupations that can be performed remotely also pay better,
as shown in table tab:Earning {[}TODO{]}. The same pattern is also found
when we split the data by occupational group (Figure figure:earning
{[}TODO{]}). However, we see two exceptions: for academic professions
where a lot of the remote-friendly occupations occur, the average wage
for non-remote workers is slightly higher. And the distribution of wages
have a much larger spread as well. The wage differential among remote
and non-remote Craft and related trades workers is much larger.

\begin{verbatim}
## \captionsetup[table]{labelformat=empty,skip=1pt}
## \begin{longtable}{lrrrr}
## \toprule
## Remote probability & \# of ocupations & \# of jobs & Average earnings & Median earnings \\ 
## \midrule
## High (>0.8) & $44$ & $358 711$ & $55 945$ & $47 763$ \\ 
## Medium (0.2 - 0.8) & $115$ & $925 782$ & $45 595$ & $42 461$ \\ 
## Low (<0.2) & $204$ & $1 498 682$ & $43 387$ & $40 983$ \\ 
## \bottomrule
## \end{longtable}
\end{verbatim}

We simplify the remote probability further for the moment by converting
it to a binary classification, simply rounding the probability to an
integer (0 or 1). This allows us to present a more comprehensible view
of the wage differences between remote- and non-remote jobs.

\includegraphics{figs/unnamed-chunk-2.pdf}

\hypertarget{variation-of-prevalence-of-remote-feasibility-across-different-regions}{%
\subsection{Variation of prevalence of remote feasibility across
different
regions}\label{variation-of-prevalence-of-remote-feasibility-across-different-regions}}

The geographic location of jobs have been a point of interest for years,
amid both pressure for workers to centralize and specialize, and fears
of increased inequality between cities and rural areas. Figure
figure:geo{[}TODO{]} shows the percentage of workers can work from home
in norway. As we expected, cities have a higher share of remote-friendly
jobs, which may be fortunate given the need for social distancing. The
pattern looks clear, especially in the area surrounding Oslo but also
the other cities like Bergen, Trondheim and Stavanger seems to stand out
on the map.

\includegraphics{figs/annotations_syss_map.pdf}

\begin{verbatim}
## \captionsetup[table]{labelformat=empty,skip=1pt}
## \begin{longtable}{lrrr}
## \toprule
## & \multicolumn{2}{c}{Number of jobs} & \\ 
##  \cmidrule(lr){2-3}
## Municipality & Total & Remote-friendly & Percent remote friendly \\ 
## \midrule
## Oppegård & $12 964$ & $5 814$ & $44,9\%$ \\ 
## Bærum & $60 596$ & $26 935$ & $44,4\%$ \\ 
## Asker & $28 960$ & $12 761$ & $44,1\%$ \\ 
## Oslo & $344 149$ & $147 810$ & $42,9\%$ \\ 
## Frogn & $7 308$ & $3 134$ & $42,9\%$ \\ 
## \bottomrule
## \end{longtable}
\end{verbatim}

\begin{verbatim}
## \captionsetup[table]{labelformat=empty,skip=1pt}
## \begin{longtable}{lrrr}
## \toprule
## & \multicolumn{2}{c}{Number of jobs} & \\ 
##  \cmidrule(lr){2-3}
## Municipality & Total & Remote-friendly & Percent remote friendly \\ 
## \midrule
## Båtsfjord & $1 023$ & $280$ & $27,4\%$ \\ 
## Torsken & $373$ & $103$ & $27,6\%$ \\ 
## Stranda & $2 318$ & $661$ & $28,5\%$ \\ 
## Frøya & $2 494$ & $711$ & $28,5\%$ \\ 
## Sørfold & $842$ & $243$ & $28,9\%$ \\ 
## \bottomrule
## \end{longtable}
\end{verbatim}

Table tab:5 most {[}TODO{]} and table tab:5 least{[}TODO{]} list the top
and bottom 5 municipalities in terms of remote friendly jobs in Norway.
We see that there are large heterogenity across different regions.
Interpolating from the annotations we estimate that 42 percent of the
jobs in Oslo can be done from home. On the other end of the spectrum, in
municipalities like Båtsfjord less than a quarter of the jobs can be
done remotely.

\includegraphics{figs/remote_density.pdf}

By introducing a measure of urbanness, we can analyze the relationship
more formally. We use population per square km as a proxy for urbanness.
From figure {[}TODO{]}figure:pop\_density, we can see a clear
correlation between \texttt{urbanness}, or population pr km\$\^{}2, and
the availability of remote-friendly jobs. Denser locations imply greater
risks of Covid-19 spread, but this increased risk may be mitigated by
better opportunities for remote work.

\hypertarget{european-results}{%
\section{European results}\label{european-results}}

Employment by first-level ISCO codes are published by the european
statistical agency Eurostat, and with a few assumptions these data can
be used to estimate remote possibilities in the EU.

Due to the lack of 4-digit ISCO codes from Eurostat, we use the
aggregates from Norwegian data as an estimate for remote percentages for
1-digit groups.

The countries with the highest share of remote-eligible jobs are rich
countries, with Switzerland and Luxembourg topping the list.

\begin{verbatim}
## \captionsetup[table]{labelformat=empty,skip=1pt}
## \begin{longtable}{lrrr}
## \toprule
## & \multicolumn{2}{c}{Number of jobs (thousand)} & \\ 
##  \cmidrule(lr){2-3}
## Country & Total & Remote-friendly & Percent \\ 
## \midrule
## Switzerland & $4 477$ & $1 795$ & $40.10\%$ \\ 
## United Kingdom & $30 943$ & $12 208$ & $39.45\%$ \\ 
## Norway & $2 565$ & $1 000$ & $38.98\%$ \\ 
## Luxembourg & $267$ & $104$ & $38.83\%$ \\ 
## Malta & $233$ & $90$ & $38.50\%$ \\ 
## Sweden & $4 888$ & $1 877$ & $38.39\%$ \\ 
## \bottomrule
## \end{longtable}
\end{verbatim}

On the other side, the countries with the lowest share of
remote-eligible jobs are mostly poorer countries in southeast europe.

\begin{verbatim}
## \captionsetup[table]{labelformat=empty,skip=1pt}
## \begin{longtable}{lrrr}
## \toprule
## & \multicolumn{2}{c}{Number of jobs (thousand)} & \\ 
##  \cmidrule(lr){2-3}
## Country & Total & Remote-friendly & Percent \\ 
## \midrule
## Romania & $8 312$ & $2 208$ & $26.57\%$ \\ 
## Turkey & $27 882$ & $7 741$ & $27.76\%$ \\ 
## North Macedonia & $742$ & $207$ & $27.95\%$ \\ 
## Serbia & $2 671$ & $798$ & $29.86\%$ \\ 
## Hungary & $4 398$ & $1 343$ & $30.54\%$ \\ 
## Bulgaria & $3 046$ & $934$ & $30.68\%$ \\ 
## \bottomrule
## \end{longtable}
\end{verbatim}

\includegraphics{figs/unnamed-chunk-6.pdf}

\hypertarget{validating-results-against-job-ads}{%
\section{Validating results against
job-ads}\label{validating-results-against-job-ads}}

The job ads can not be compared directly to the annotated ISCO-08 data,
but one of the comparisons we are able to do is the relative frequency
of remote possibilities across occupational groups. Some job ads mention
remote work, and we expect that remote possibilities are mentioned more
often for occupations where remote possibilities are an option. We can
compare this frequency with the number of remote-friendly occupations
within each broader group (table tab:relative{[}TODO{]}). We also
compute the composition of remote-friendly jobs across the occupational
groups, giving us a measure that is directly comparable.

\begin{verbatim}
## \captionsetup[table]{labelformat=empty,skip=1pt}
## \begin{longtable}{lrrr}
## \toprule
## & \multicolumn{2}{c}{Relative remote frequency} & \\ 
##  \cmidrule(lr){2-3}
## Occupational group & Annotations & Job ads & Difference \\ 
## \midrule
## Managers & $4,1\%$ & $4,5\%$ & $-0,4\%$ \\ 
## Academics & $66,5\%$ & $49,9\%$ & $16,6\%$ \\ 
## Assc. professionals & $13,1\%$ & $32,2\%$ & $-19,1\%$ \\ 
## Clerics & $3,5\%$ & $5,3\%$ & $-1,8\%$ \\ 
## Service workers & $6,4\%$ & $6,9\%$ & $-0,6\%$ \\ 
## Agricultural workers & $0,0\%$ & $0,0\%$ & $0,0\%$ \\ 
## Craft workers & $5,6\%$ & $0,9\%$ & $4,7\%$ \\ 
## NA & $0,5\%$ & $0,1\%$ & $0,3\%$ \\ 
## NA & $0,2\%$ & $0,0\%$ & $0,2\%$ \\ 
## \bottomrule
## \end{longtable}
\end{verbatim}

The notable discrepancies for the groups \texttt{Academics} and
\texttt{Technicians\ and\ associate\ professionals} have a curious
symmetry. Both of these groups require higher education or similar skill
level, but Technicians and associate professionals are more vocational.
There are many possible explanations for such differences, but for now
it will suffice to conclude that one or more of the assumptions made is
violated to some extent. Still the correlation is decent considering the
spuriousness of the data.

In the appendix, we include some more discussions on this issue. It
contains also a method we have developed to utilize job announcement
data to correct for possible classification bias in the annotation
labels. The magnitudes of the estimated biases are relatively small.
Detailed estimates can also be found in appendix.

\hypertarget{related-work}{%
\section{Related work}\label{related-work}}

There are several very recent analyses that study the remote feasibility
of jobs: \citep{Dingel2020}, \citep{Brynjolfsson2020} and
\citep{Hensivk2020} for the United States, \citep{Alipour2020} for
Germany, and \citep{Barbieri2020} for Italy. Unlike our study, they
mostly rely on different types of survey data. However, the results can
be compared given the similar labor market institutions and labor force
skill levels.

In Norway, as mentioned in section sec:introduction{[}TODO{]}, there are
two surveys related to our study. The results from the labor force
survey was broadly similar to what we obtained with some differences. In
the labor force survey, 71 per cent of managers responded that they had
opportunity to work from home at times, much higher than the results
here. This is likely attributable to the distinction mentioned above:
Being able to perform \emph{some} part of your job remotely does not
mean that the job can be performed \emph{primarily} remotely. The second
survey by TØI was carried out after the Covid-19 outbreak in Norway. It
reports an higher percentage of jobs (nearly 50\%) are done from home
these days. However, it is also reported considerable efficiency loss
connoted to the home office arrangement. It may be that workers are
stretching the limits of what can be effectively accomplished from home,
or perhaps the annotations are slightly conservative. Note that although
annotations were collected after the Covid-19 outbreak, it is hard to
know whether the annotators had the pandemic in mind when annotating.

\hypertarget{conclusion}{%
\section{Conclusion}\label{conclusion}}

The sudden question of remote work highlights the need to expand our
knowledge of occupations and their contents. In this paper we study the
remote feasibility for different occupations in Norway. This analysis
sheds more lights on a fundamental problem in the labor market. And more
importantly it also provides useful knowledge for decision makers to
evaluate potential social policies to combat the Covid-19 pandemic.
Norwegian government will re-open primary schools and kindergartens
soon. However, opening a school where majority of the parents can work
from home would have rather different implications on potential virus
spread, if compared with opening a school where majority of parents
don't have this option. In addition, occupations with different remote
feasibility will not be hit as hard, so these estimates would be useful
when one want to assess potential economic impacts of the pandemic.

On the other hand, for National Statistical Institutions in Europe, the
most natural option would be to expand the existing ISCO ontology with
this data. This analysis is an attempt to combine the conventional and
unconventional sources for statisical and research purpose. The results
we have found also suggest that alternative approaches to collecting
such information is feasible. The ISCO-08 documentation includes more
detailed descriptions of the jobs than what we made use of in this
paper. The complete list includes lists of tasks commonly performed,
which may further nuance what parts of a job can actually be performed
from home. The ISCO ontology may also serve this purpose.Although there
were some discrepancies between the job ads and the annotations, there
is clearly a pattern, and the differences between such sources should be
explored. From an economic perspective this may tell us something about
employer preferences related to hiring, and from the perspective of a
National Statistical Institution such sources may, if calibrated
correctly, provide further information about the labor market and
working conditions. The possibilities are not limited to the question of
working from home. Job ads are a great resource for describing actual
jobs and what they entail, although the picture the advertisements paint
might be a little rosy.

\newpage
\singlespacing 
\bibliography{wfh.bib}
\end{document}